\chapter{入門}

\section{相依套件}

Spring Boot 專案,將以下相依套件加到 \textit{pom.xml}。實際上,Spring Boot 使用的 Validation 為 \textit{org.{\allowbreak}hibernate.{\allowbreak}validator}~\footnote{\url{https://github.com/spring-projects/spring-boot/blob/main/spring-boot-project/spring-boot-starters/spring-boot-starter-validation/build.gradle}}。

\begin{lstlisting}
<dependency>
  <groupId>org.springframework.boot</groupId>
  <artifactId>spring-boot-starter-validation</artifactId>
</dependency>
\end{lstlisting}

或 Maven 專案,將以下相依套件加到 \textit{pom.xml}。

\begin{lstlisting}
<dependency>
  <groupId>org.glassfish</groupId>
  <artifactId>jakarta.el</artifactId>
  <version>4.0.0</version>
</dependency>
<dependency>
  <groupId>org.hibernate.validator</groupId>
  <artifactId>hibernate-validator</artifactId>
  <version>7.0.1.Final</version>
</dependency>
\end{lstlisting}

以下是方便測試的套件:

\begin{lstlisting}
<dependency>
  <groupId>org.hibernate.validator</groupId>
  <artifactId>hibernate-validator-test-utils</artifactId>
  <version>7.0.1.Final</version>
  <scope>test</scope>
</dependency>
\end{lstlisting}

\section {套用限制}

汽車物件為範例:

\begin{lstlisting}
import lombok.Data;
import lombok.RequiredArgsConstructor;

import jakarta.validation.constraints.Min;
import jakarta.validation.constraints.NotNull;
import jakarta.validation.constraints.Size;


@Data
@RequiredArgsConstructor
public class Car {
    @NotNull
    private String manufacturer;

    @NotNull
    @Size(min = 2, max = 14)
    private String licensePlate;

    @Min(2)
    private int seatCount;
}

\end{lstlisting}

\begin{itemize}
\item \texttt{manufacturer} 不能為 \textit{null}。
\item \texttt{licensePlate} 不能為 \textit{null} 並且必須介於 2 至 14 字元長度。
\item \texttt{seatCount} 至少為 2。
\end{itemize}

\section{檢驗限制}

使用物件 \texttt{Validator} 驗證物件是否破壞限制,以下是單元測試。

\begin{lstlisting}
import java.util.Set;
import jakarta.validation.ConstraintViolation;
import jakarta.validation.Validation;
import jakarta.validation.Validator;
import jakarta.validation.ValidatorFactory;

import org.junit.BeforeClass;
import org.junit.Test;

import static org.junit.Assert.assertEquals;

public class CarTest {
  private static Validator validator;

  @BeforeClass
  public static void setUpValidator() {
    ValidatorFactory factory = Validation.buildDefaultValidatorFactory();
    validator = factory.getValidator();
  }

  @Test
  public void manufacturerIsNull() {
    Car car = new Car( null, "DD-AB-123", 4 );

    Set<ConstraintViolation<Car>> constraintViolations =
      validator.validate( car );

    assertEquals( 1, constraintViolations.size() );
    assertEquals( "must not be null",
      constraintViolations.iterator().next().getMessage() );
  }

  // 其他欄位的驗證忽略。

  @Test
  public void carIsValid() {
    Car car = new Car( "Morris", "DD-AB-123", 2 );

    Set<ConstraintViolation<Car>> constraintViolations =
      validator.validate( car );

    assertEquals( 0, constraintViolations.size() );
  }
}
\end{lstlisting}

在 \texttt{setUpValidator} 方法,測試類別最開始將透過 \texttt{ValidatorFactory} 生成 \texttt{Validator} 物件。\texttt{Validator} 物件是執行緒安全(thread-safe)且可重複使用,所以可以它可以宣告呈靜態物件,並且在各測試重複使用。

\texttt{validate} 方法回傳 \texttt{ConstraintViolation} 物件,我們可以透過迭代的方式,將取得破壞限制的欄位訊息。若沒有任何欄位破壞限制,方法回傳空集合。

這裡只使用到 \textit{jakarta.validation}~\footnote{在 Jakarta EE 9 版本的時候,javax.* 命名空間被改名成 jakarta.*。Jakarta EE 的前身是 Java EE;而 Java EE (Java EE) 的前身 J2EE。},它是 Bean Validation API 的一部份,並沒有直接與 Hibernate Validator 有直接耦合的情況,因此可以提高程式碼的可移植性。