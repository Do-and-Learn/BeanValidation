\chapter{宣告與驗證 Bean 的限制}

\section{宣告 Bean 的限制}

Jakarta Bean Validation 使用 Java 標記定義限制。限制分成四類:

\begin{enumerate}
\item \textit{欄位層級的限制}(\textit{field-level constraints}): 將限制標籤標記在欄位宣告上。
\item \textit{屬性層級的限制}(\textit{property-level constraints}): 將限制標記在存取方法上(Getter)。
\item \textit{容器元素限制}(\textit{container element constraints}): 將限制標記在 Generic Type 上。
\item \textit{類別限制}(\textit{class constraints})
\end{enumerate}

\subsection{欄位層級的限制}

\begin{lstlisting}
import lombok.Data;
import lombok.RequiredArgsConstructor;

import jakarta.validation.constraints.NotNull;
import jakarta.validation.constraints.AssertTrue;

@Data
@RequiredArgsConstructor
public class Car {
  @NotNull
  private String manufacturer;

  @AssertTrue
  private boolean isRegistered;
}
\end{lstlisting}

\textit{驗證引擎}(\textit{Validation Engine}) 直接存取物件欄位的值,進行驗證,不會透過 Getter 方法存取數值。即使欄位的存取類型被設定成 \textit{private} 也無妨~\footnote{沒意外應該是透過 relection 存取數值?}。不過,\textbf{不支援靜態欄位的限制}。

接下來的屬性層級的限制,它將限制標籤標記在 Getter 上。不過,像本範例使用 \texttt{@Data} 幫助生成 Getter 的情況下,則無法使用屬性層級的限制,只能使用欄位層級的限制。

\subsection{屬性階層的限制}

\begin{lstlisting}
import lombok.Setter;
import lombok.RequiredArgsConstructor;

import jakarta.validation.constraints.NotNull;
import jakarta.validation.constraints.AssertTrue;

@RequiredArgsConstructor
@Setter
public class Car {

  private String manufacturer;

  private boolean isRegistered;

  @NotNull
  public String getManufacturer() {
    return manufacturer;
  }

  @AssertTrue
  public boolean isRegistered() {
    return isRegistered;
  }
}
\end{lstlisting}

驗證引擎(Validation Engine) 將透過 Getter 存取物件的數值進行驗證。

建議一個欄位的 欄位層級的限制 與 屬性階層的限制 擇一使用,否則欄位將被驗證兩次。

\subsection{容器元素限制}

標記在通用型態(Generic Type)上的限制。

\subsubsection{\texttt{Iterable}}

\begin{lstlisting}
import java.util.HashSet;
import java.util.Set;

public class Car {
  private Set<@ValidPart String> parts = new HashSet<>();

  public void addPart(String part) {
    parts.add( part );
  }
}
\end{lstlisting}

\texttt{@ValidPart} 限制 \texttt{parts} 集合中的元素。在 Hibernate Validator 6 以前,需要寫成 \texttt{@ValidPart private Set<String> parts},雖然此語法仍然支援,但不建議使用。

其他宣告 Generic Type 的限制同此範例標記在 Type 之前,其他容器如\texttt{List}、\texttt{Map}、\texttt{java.util.Optional} 可以參考官方教學文件~\cite{hibernate-validator-doc}。

\subsubsection{自訂容器的型態}

\begin{lstlisting}
public class Car {
  private GearBox<@MinTorque(100) Gear> gearBox;
}
\end{lstlisting}

\texttt{@MinTorque(100)} 限制 \texttt{gearBox} 物件的齒輪(Gear)最小扭力必須為 100。實作參考第~\ref{sec:code}小節程式碼。

\subsubsection{巢狀容易的限制}

\begin{lstlisting}
public class Car {
  private Map<@NotNull Part, List<@NotNull Manufacturer>> partManufacturers =
    new HashMap<>();
}
\end{lstlisting}

\texttt{@NotNull} 限制了 \texttt{partManufacturers} 物件的 索引(key)值 以及對應值的清單中的元素。

\subsection{類別限制}

\begin{lstlisting}
@ValidPassengerCount
public class Car {
  private int seatCount;
  private List<Person> passengers;
}
\end{lstlisting}

\texttt{@ValidPassengerCount} 限制 \texttt{Car} 物件的乘客數必須是有效的。類別限制不是用於檢驗單一欄位,而是用於檢驗一組相關的欄位。以此例來說,座位數以及乘客數才能夠決定乘客數是否有效。

\subsection{逐級驗證(Cascaded Validation)}

\begin{lstlisting}
@Data
public class Car {
  @NotNull
  @Valid
  private Person driver;
}
\end{lstlisting}

\texttt{@Valid} 將使得物件進行\textit{逐級驗證}(\textit{Cascaded Validation}),他將首先驗證 \texttt{driver} 不能是 \textit{null},並且接著驗證 \texttt{Person} 物件,驗證 \texttt{name} 不能是 \textit{null}。若沒有標記 \texttt{@Valid},那麼驗證程序就會停止在確認 \texttt{driver} 不是 \textit{null}。

\begin{lstlisting}
@Data
public class Person {
  @NotNull
  private String name;
}
\end{lstlisting}

\section{驗證 Bean 的限制}

\texttt{Validator} 物件用於驗證限制,\texttt{ValidatorFactory} 物件用於產生 \texttt{Validator} 物件。以下程式碼用於產生預設的 \texttt{Validator}。

\begin{lstlisting}
ValidatorFactory factory = Validation.buildDefaultValidatorFactory();
Validator validator = factory.getValidator();
\end{lstlisting}

\texttt{Validator} 提供三個驗證的方法:

\begin{enumerate}
\item \texttt{validate}
\item \texttt{validateProperty}
\item \texttt{validateValue}
\end{enumerate}

這三個方法皆回傳 \texttt{Set<ConstraintViolation>}。若沒有破壞任何限制,則回傳空集合。反之,每一筆 \texttt{ConstraintViolation} 都代表一筆破壞的限制。

\begin{lstlisting}
Car car = new Car( null, true );

Set<ConstraintViolation<Car>> constraintViolations = validator.validate( car );

assertEquals( 1, constraintViolations.size() );
assertEquals( "must not be null",
              constraintViolations.iterator().next().getMessage() );
\end{lstlisting}

檢驗物件 \texttt{car} 是否有違反任何限制。

\begin{lstlisting}
Car car = new Car( null, true );

Set<ConstraintViolation<Car>> constraintViolations = validator.validateProperty(
  car, "manufacturer"
);

assertEquals( 1, constraintViolations.size() );
assertEquals( "must not be null",
              constraintViolations.iterator().next().getMessage() );
\end{lstlisting}

檢驗物件 \texttt{car} 的屬性 \texttt{manufacturer} 是否違反限制。

\begin{lstlisting}
Set<ConstraintViolation<Car>> constraintViolations = validator.validateValue(
  Car.class, "manufacturer", null
);

assertEquals( 1, constraintViolations.size() );
assertEquals( "must not be null",
              constraintViolations.iterator().next().getMessage() );
\end{lstlisting}

當物件 \texttt{Car} 的屬性 \texttt{manufacturer} 為 \textit{null} 時,是否違反限制。
\texttt{@Valid} 標記不適用在 \texttt{validateProperty} 與 \texttt{validateValue}。

\subsection{\texttt{ConstraintViolation}}

方法:

\begin{tabular}{|l|l|l|}
\hline
\textbf{方法} & \textbf{說明} & \textbf{範例} \\ \hline
\texttt{getMessage} & 違反限制的錯誤訊息。 & must not be null \\ \hline
\texttt{getMessageTemplate} & 違反限制的錯誤訊息的索引。 & {…​ NotNull.message} \\ \hline
\texttt{getRootBean} & 導致錯誤的根源 & car \\ \hline
\texttt{getRootBeanClass} & 導致錯誤的根源類別 & Car.class \\ \hline
\texttt{getLeafBean} & 導致錯誤的物件 & car\\ \hline
\texttt{getPropertyPath} & 從根源類別到破壞限制欄位的路徑 & manufacturer \\ \hline
\texttt{getInvalidValue} & 破壞限制的數值 & null \\ \hline
\texttt{getConstraintDescriptor} & 限制的描述 & \texttt{@NotNull} 的描述 \\ \hline
\end{tabular}

\section{內建的限制標記}

\subsection{Jakarta Bean Validation 限制}

這些限制都是欄位/屬性階級的限制,而非類別的限制。

\vspace{10pt}

\noindent\textbf{任何型態的限制}

\begin{itemize}
\item \texttt{@NotNull}: 數值不能為 \textit{null}。 (Hibernate metadata)
\item \texttt{@Null}: 數值必須為 \textit{null}。
\end{itemize}

\noindent\textbf{布林型態的限制}

\begin{itemize}
\item \texttt{@AssertFalse}: 數值必須為 \textit{false}。
\item \texttt{@AssertTrue}: 數值必須為 \textit{true}。
\end{itemize}

\noindent\textbf{數值的限制}\vspace{10pt}

應用在 BigDecimal, BigInteger, byte, short, int, long (Hibernate Validator 額外支援: CharSequence, Number, javax.money.MonetaryAmount) 型態的限制。

\begin{itemize}
\item \texttt{@Max(value=)}: 限制數值的最大值。(Hibernate metadata)
\item \texttt{@Min(value=)}: 限制數值的最小值。(Hibernate metadata)
\item \texttt{@Positive}: 限制數值必須為正數。(0 視為無效值)
\item \texttt{@PositiveOrZero}: 限制數值必須為正數。
\item \texttt{@Negative}: 限制數值必須為負數。(0 視為無效值)
\item \texttt{@NegativeOrZero}: 限制數值必須為負數。
\end{itemize}

應用在 BigDecimal, BigInteger, \textbf{CharSequence}, byte, short, int, long (Hibernate Validator 額外支援: Number, javax.money.MonetaryAmount) 型態的限制。

\begin{itemize}
\item \texttt{@DecimalMax(value=, inclusive=)}: 限制數值的最大值。
\item \texttt{@DecimalMin(value=, inclusive=)}: 限制數值的最小值。
\item \texttt{@Digits(integer=, fraction=)}: 限制數值的整數與小數位數。(Hibernate metadata)
\end{itemize}

\noindent\textbf{日期時間的限制}\vspace{10pt}

應用在 java.util.Date, java.util.Calendar, java.time.Instant, java.time.LocalDate, java.time.LocalDateTime, java.time.LocalTime, java.time.MonthDay, java.time.OffsetDateTime, java.time.OffsetTime, java.time.Year, java.{\allowbreak}time.YearMonth, java.time.ZonedDateTime, java.time.chrono.HijrahDate, java.time.chrono.JapaneseDate,{\linebreak} java.time.chrono.MinguoDate, java.time.chrono.ThaiBuddhistDate (Hibernate Validator 額外支援: ReadablePartial, ReadableInstant) 型態的限制。

\begin{itemize}
\item \texttt{@Future}: 限制時間必須是未來的時間。
\item \texttt{@FutureOrPresent}: 限制時間必須是現在或未來的時間。
\item \texttt{@Past}: 限制時間必須是過去的時間。
\item \texttt{@PastOrPresent}: 限制時間必須是現在或過去的時間。
\end{itemize}

\noindent\textbf{字串的限制}

\vspace{10pt}

應用在 CharSequence 型態的限制。

\begin{itemize}
\item \texttt{@Email}: 限制數值必須為有效的郵件地址。
\item \texttt{@NotBlank}: 限制數值內容不為 \textit{null} 以及去頭尾空白的長度為 0。
\item \texttt{@Pattern(regex=, flags=)}: 限制數值必須符合正規表示式。
\end{itemize}

應用在 CharSequence, Collection, Map 與 陣列 型態的限制。

\begin{itemize}
\item \texttt{@NotEmpty}: 數值內容為空。
\item \texttt{@Size(min=, max=)}: 數值的容量必須介於最大與最小值之間(含)。(Hibernate metadata)
\end{itemize}

\subsection{額外的限制}

\noindent\textbf{任何型態的限制}\vspace{10pt}

\begin{itemize}
\item \texttt{@ScriptAssert(lang=, script=, alias=, reportOn=)}
\end{itemize}

\noindent\textbf{字串的限制}\vspace{10pt}

應用在 CharSequence 型態的限制。

\begin{itemize}
\item \texttt{@CreditCardNumber(ignoreNonDigitCharacters=)}: 用以檢查使用者是否可能誤植卡號,但不能保證卡號是有效的。
\item \texttt{@EAN}: 檢查字串是否為有效的 EAN barcode。
\item \texttt{@ISBN}: 檢查字串是否為有效的 ISBN。
\item \texttt{@Length(min=, max=)}~\footnote{應使用標準的限制標籤: \texttt{@Size}}: 數值必須介於最大與最小值之間(含)。(Hibernate metadata)
\item \texttt{@CodePointLength(min=, max=, normalizationStrategy=)}~\footnote{\url{https://hibernate.atlassian.net/browse/HV-1496}}
\item \texttt{@LuhnCheck(startIndex= , endIndex=, checkDigitIndex=, ignoreNonDigitCharacters=)}: 限制字串須符合 Luhn 檢查碼。字串被設計成 Luhn 算法,主要是為了防止使用者輸入資料時,誤植出錯。
\item \texttt{@Mod10Check(multiplier=, weight=, startIndex=, endIndex=, checkDigitIndex=, ignoreNonDigitCharacters=)}: 限制字串須符合 Mod 10 檢查碼。
\item \texttt{@Mod11Check(threshold=, startIndex=, endIndex=, checkDigitIndex=, ignoreNonDigitCharacters=, treatCheck10As=, treatCheck11As=)}: 限制字串須符合 Mod 11 檢查碼。
\item \texttt{@Normalized(form=)}
\item \texttt{@URL(protocol=, host=, port=, regexp=, flags=)}
\end{itemize}

\noindent\textbf{數值的限制}\vspace{10pt}

應用在 BigDecimal, BigInteger, CharSequence, byte, short, int, long 型態的限制。

\begin{itemize}
\item \texttt{@Range(min=, max=)}
\end{itemize}

\noindent\textbf{期間的限制}\vspace{10pt}

應用在 java.time.Duration 型態的限制。

\begin{itemize}
\item \texttt{@DurationMax(days=, hours=, minutes=, seconds=, millis=, nanos=, inclusive=)}: 限制最大期間。
\item \texttt{@DurationMin(days=, hours=, minutes=, seconds=, millis=, nanos=, inclusive=)}: 限制最小期間。
\end{itemize}

\noindent\textbf{集合的限制}\vspace{10pt}

應用在 Collection 型態的限制。

\begin{itemize}
\item \texttt{@UniqueElements}
\end{itemize}

\noindent\textbf{其他的限制}\vspace{10pt}

應用在 javax.money.MonetaryAmount 型態的限制。

\begin{itemize}
\item \texttt{@Currency(value=)}: 限制必須是有效的幣值單位。
\end{itemize}