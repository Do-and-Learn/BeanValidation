\chapter{Method 限制的宣告與驗證}

限制 Method 的 pre-condition 與 post-condition。不適用於靜態的 Method。

\section{宣告 Method 的限制}

\subsection{參數的限制}

\begin{lstlisting}[escapeinside={`}{`}]
public class RentalStation {
  public RentalStation(`\color{blue}@NotNull` String name) {
    //...
  }

  public void rentCar(
    `\color{blue}@NotNull` Customer customer,
    `\color{blue}@NotNull` `\color{blue}@Future` Date startDate,
    `\color{blue}@Min(1)` int durationInDays) {
    //...
  }
}
\end{lstlisting}

這些標記在參數上的限制,並不會在呼叫的時候自動地驗證。

\begin{lstlisting}[escapeinside={`}{`}]
public class Car {
  `\color{blue}@LuggageCountMatchesPassengerCount(piecesOfLuggagePerPassenger = 2)`
  public void load(List<Person> passengers, List<PieceOfLuggage> luggage) {
    //...
  }
}
\end{lstlisting}

這是跨參數限制的範例。

\begin{lstlisting}[escapeinside={`}{`}]
public class Garage {
  @ELAssert(expression = "...", `\color{blue}validationAppliesTo = ConstraintTarget.PARAMETERS`)
    public Car buildCar(List<Part> parts) {
      //...
      return null;
    }

  @ELAssert(expression = "...", `\color{blue}validationAppliesTo = ConstraintTarget.RETURN\_VALUE`)
  public Car paintCar(int color) {
    //...
    return null;
  }
}
\end{lstlisting}

\texttt{validationAppliesTo} 指定限制要套用在參數上還是回傳值上。以下狀況,限制將會自動參考要限制的部分。回傳型態為 \texttt{void},限制將套用至參數;無參數時,限制相套用至回傳值。

\subsection{回傳值的限制}

\begin{lstlisting}[escapeinside={`}{`}]
public class RentalStation {
  `\color{blue}@ValidRentalStation`
  public RentalStation() {
    //...
  }

  `\color{blue}@NotNull`
  `\color{blue}@Size(min = 1)`
  public List<`\color{blue}@NotNull Customer>` getCustomers() {
    //...
    return null;
  }
}
\end{lstlisting}

新建的 \texttt{RentalStation} 物件必須滿足 \texttt{ValidRentalStation} 限制。\texttt{getCustomers()} 的回傳值不能為 \textit{null}並至少含一個元素,且元素不能為 \textit{null}。

\subsection{逐級驗證(Cascaded validation)}

\begin{lstlisting}[escapeinside={`}{`}]
public class Garage {

  @NotNull
  private String name;

  `\color{blue}@Valid`
  public Garage(String name) {
    this.name = name;
  }

  public boolean checkCar(`\color{blue}@Valid` @NotNull Car car) {
    //...
    return false;
  }
}
\end{lstlisting}

建構式標記 \texttt{@Valid} 表對回傳的物件應進行逐級驗證。以及 \texttt{checkCar} 的參數 \texttt{car} 進行逐級驗證。

\subsection{方法限制的繼承}

\begin{itemize}
\item 子類別可能無法加強前置條件的限制。
\item 子類別無法減弱後置條件的限制。
\end{itemize}

\begin{lstlisting}
public interface Vehicle {
    void drive(@Max(75) int speedInMph);
}

public class Car implements Vehicle {
    @Override
    public void drive(@Max(55) int speedInMph) {
        //...
    }
}
\end{lstlisting}

{\color{red}如果介面的方法沒有宣告限制的話,其衍生類別也無法標記限制?}

\begin{lstlisting}
public interface Vehicle {
    void drive(@Max(75) int speedInMph);
}

public interface Car {
    void drive(int speedInMph);
}

public class RacingCar implements Car, Vehicle {
    @Override
    public void drive(int speedInMph) {
        //...
    }
}
\end{lstlisting}

{\color{red}\texttt{RacingCar\#drive} 覆載 \texttt{Vehicle\#drive} 與 \texttt{Car\#drive},所以 \texttt{RacingCar\#drive} 的限制就失效了?}

\begin{lstlisting}[escapeinside={`}{`}]
public interface Vehicle {
    `\color{blue}@NotNull`
    List<Person> getPassengers();
}

public class Car implements Vehicle {
    @Override
    `\color{blue}@Size(min = 1)`
    public List<Person> getPassengers() {
        //...
        return null;
    }
}
\end{lstlisting}

\texttt{Car\#getPassengers} 將套用 \texttt{@NotNull} 與 \texttt{@Size} 兩個限制。

\section{驗證 Method 的限制}

這裡驗證需要仰賴 AOP 或者 Proxy,因為極度不 type-safe。對於程式的維護性不友善,個人不是很建議以下的作法。

\begin{lstlisting}
public class Car {

  public Car(@NotNull String manufacturer) {}

  public void drive(@Max(75) int speedInMph) {}

  @Size(min = 1)
  public List<Object> getPassengers() {
    return Collections.emptyList();
  }
}
\end{lstlisting}

驗證 Method 的限制需透過 \texttt{ExecutableValidator} 物件。以下是測試程式碼。

\begin{lstlisting}[basicstyle=\linespread{1}\ttfamily\footnotesize]
private ExecutableValidator executableValidator;

@BeforeMethod
public void setup() {
  ValidatorFactory factory = Validation.buildDefaultValidatorFactory();
  executableValidator = factory.getValidator().forExecutables();
}

@Test
public void testInvalidConstructor() throws NoSuchMethodException {
  Set<ConstraintViolation<Car>> violations = executableValidator
    .validateConstructorParameters(Car.class.getConstructor(String.class), new Object[]{null});
  assertThat(violations).containsOnlyViolations(
    violationOf(NotNull.class).withPropertyPath(pathWith().constructor(Car.class).parameter("arg0", 0))
  );
}

@Test
public void testValidConstructor() throws NoSuchMethodException {
  Set<ConstraintViolation<Car>> violations = executableValidator
    .validateConstructorParameters(Car.class.getConstructor(String.class), new Object[]{"Toyota"});
  assertNoViolations(violations);
}

@Test
public void invalidParameter() throws NoSuchMethodException {
  Car car = new Car("Toyota");

  Set<ConstraintViolation<Car>> violations = executableValidator.validateParameters(
    car, Car.class.getDeclaredMethod("drive", int.class), new Object[]{100}
  );

  assertThat(violations).containsOnlyViolations(
    violationOf(Max.class).withPropertyPath(pathWith().method("drive").parameter("arg0", 0))
  );
}

@Test
public void validParameter() throws NoSuchMethodException {
  Car car = new Car("Toyota");

  Set<ConstraintViolation<Car>> violations = executableValidator.validateParameters(
    car, Car.class.getDeclaredMethod("drive", int.class), new Object[]{75}
  );

  assertNoViolations(violations);
}

@Test
public void invalidReturnValue() throws NoSuchMethodException {
  Car car = new Car("Toyota");
  Set<ConstraintViolation<Car>> violations = executableValidator.validateReturnValue(
    car, Car.class.getDeclaredMethod("getPassengers"), car.getPassengers()
  );

  assertThat(violations).containsOnlyViolations(
    violationOf(Size.class).withPropertyPath(pathWith().method("getPassengers").returnValue())
  );
}

@Test
public void validReturnValue() throws NoSuchMethodException {
  Car car = new Car("Toyota");
  Set<ConstraintViolation<Car>> violations = executableValidator.validateReturnValue(
    car, Car.class.getDeclaredMethod("getPassengers"), Collections.singleton(new Object())
  );

  assertNoViolations(violations);
}
\end{lstlisting}

\section{內建 Method 的限制}

\begin{lstlisting}
public class Car {

  @ParameterScriptAssert(
    lang = "javascript",
    script = "luggage.size() <= passengers.size() * 2")
  public void load(List<Person> passengers, List<PieceOfLuggage> luggage) {
    //...
  }
}
\end{lstlisting}