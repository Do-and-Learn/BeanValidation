\chapter{限制的分群}

\section{定義群組}

\begin{lstlisting}
public interface DriverChecks {}

@Data
public class Driver {
  @NotNull
  private String name;

  @Min(value = 18, groups = DriverChecks.class)
  private int age;

  @AssertTrue(groups = DriverChecks.class)
  private boolean hasDrivingLicense;
}
\end{lstlisting}

沒有給予 \textit{group} 參數的限制,則屬於 group \texttt{jakarta.validation.groups.Default}。

程式碼中 \texttt{Default} group 的有 \texttt{Driver.name};\texttt{DriverChecks} group 的有 \texttt{Driver.age}、\texttt{Driver.hasDriving{\break}License}。

測試程式碼如下:

\begin{lstlisting}
import static org.hibernate.validator.testutil.ConstraintViolationAssert.assertThat;
import static org.hibernate.validator.testutil.ConstraintViolationAssert.violationOf;

private Validator validator;
private Driver driver;

@BeforeClass
public void setUp() {
  validator = Validation.buildDefaultValidatorFactory().getValidator();
  driver = new Driver();
  driver.setAge(10);
  driver.setHasDrivingLicense(false);
}

@Test
public void defaultValidate() {
  Set<ConstraintViolation<Driver>> violations = validator.validate(driver);

  assertThat(violations).containsOnlyViolations(
    violationOf(NotNull.class)
  );
}

@Test
public void groupValidate() {
  Set<ConstraintViolation<Driver>> violations = validator.validate(
    driver, DriverChecks.class);

  assertThat(violations).containsOnlyViolations(
    violationOf(Min.class),
    violationOf(AssertTrue.class)
  );
}
\end{lstlisting}

\section{群組的繼承}

將前一小節之 \texttt{DriverChecks} 繼承預設限制群組。驗證指定的群組,將連同祖先類別的群組一併驗證。

\begin{lstlisting}
import jakarta.validation.groups.Default;

public interface DriverChecks `\color{cyan}extends Default` {}
\end{lstlisting}

測試程式碼調整如下:

\begin{lstlisting}[numbers=left, xleftmargin=1.5\parindent]
@Test
public void defaultValidate() {
  Set<ConstraintViolation<Driver>> violations = validator.validate(driver);

  assertThat(violations).containsOnlyViolations(
    violationOf(NotNull.class)
  );
}

@Test
public void groupValidate() {
  Set<ConstraintViolation<Driver>> violations =
    validator.validate(driver, DriverChecks.class);

  assertThat(violations).containsOnlyViolations(
    `\color{cyan}violationOf(NotNull.class)`,
    violationOf(Min.class),
    violationOf(AssertTrue.class)
  );
}
\end{lstlisting}

\texttt{DriverChecks} 也是 \texttt{Default},因此第 13 行驗證結果會包含 \texttt{Default} group 的驗證結果。

\section{群組順序}

\begin{lstlisting}
`\color{cyan}@GroupSequence({ Default.class, CarChecks.class, DriverChecks.class })`
public interface OrderedChecks {}

validator.validate( car, OrderedChecks.class )
\end{lstlisting}

依照 \texttt{GroupSequence} 中定義的限制順序,依序驗證限制是否被破壞。此順序不得包含循環相依,否則將引起例外 \texttt{GroupDefinitionException}。

\section{重新定義 \texttt{Default}}

\subsection{\texttt{@GroupSequence}}

\begin{lstlisting}
@Data
`\color{cyan}@GroupSequence(\{DriverChecks.class, \underline{Driver.class}\})`
public class Driver {
  @NotNull
  private String name;

  @Min(value = 18, groups = DriverChecks.class)
  private int age;

  @AssertTrue(groups = DriverChecks.class)
  private boolean hasDrivingLicense;
}
\end{lstlisting}

\texttt{@GroupSequence} 中必須包含自身類別。

\begin{lstlisting}
@Test
public void defaultValidate() {
  Set<ConstraintViolation<Driver>> violations = validator.validate(driver);

  assertThat(violations).containsOnlyViolations(
    violationOf(Min.class),
    violationOf(AssertTrue.class)
  );
}
\end{lstlisting}

因此預設的驗證就被重新定義成檢查群組 \texttt{DriverChecks} 的限制是否被破壞。而預設的群組 \texttt{@NotNull} 的限制是否被破壞則無檢查。

\subsection{\texttt{@GroupSequenceProvider}}

\begin{lstlisting}
import org.hibernate.validator.spi.group.DefaultGroupSequenceProvider;

import java.util.ArrayList;
import java.util.List;

public class DriverGroupSequenceProvider implements DefaultGroupSequenceProvider<Driver> {
  @Override
  public List<Class<?>> getValidationGroups(Driver driver) {
    List<Class<?>> defaultGroupSequence = new ArrayList<>();
    defaultGroupSequence.add(DriverChecks.class);
    defaultGroupSequence.add(Driver.class);
    return defaultGroupSequence;
  }
}

\end{lstlisting}

可以根據 \texttt{driver} 的狀態決定群組順序。

\begin{lstlisting}
@Data
`\color{cyan}@GroupSequenceProvider(DriverGroupSequenceProvider.class)`
public class Driver {
  @NotNull
  private String name;

  @Min(value = 18, groups = DriverChecks.class)
  private int age;

  @AssertTrue(groups = DriverChecks.class)
  private boolean hasDrivingLicense;
}
\end{lstlisting}

此等價於前一小節的範例。

\section{群組轉換}

\begin{lstlisting}
public class Car {
  @Valid
  `\color{cyan}@ConvertGroup(from = Default.class, to = DriverChecks.class)`
  private Driver driver;
}
\end{lstlisting}

\texttt{@ConvertGroup} 必須與 \texttt{@Valid} 一同使用。