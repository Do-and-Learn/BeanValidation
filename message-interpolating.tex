\chapter[]{生成(Interpolating)\footnote{找不到適合的 Interpolating 中文對應詞,即使英英的解釋也未能貼合此處的意涵。這裡姑且就將其作生成吧。}限制的錯誤訊息}

\textit{訊息生成}(\textit{Message interpolation})是 Jakarta Bean 限制被破壞的錯誤訊息產生的過程。本章節將說明 (1) 定義錯誤訊息、(2) 訊息內容的決定 以及 (3) 自定義錯誤訊息。

\section{預設訊息的生成}

限制被破壞的訊息來自\textit{訊息描述}(\textit{message descriptors})。每種限制透過 \texttt{message} 屬性設定其預設的訊息描述。在宣告限制時,訊息描述可以被覆寫,範例如下:

\begin{lstlisting}
@Data
public class Car {
  @NotNull(`\color{blue}message = "製造商名稱不得為 null"`)
  private String manufacturer;
}
\end{lstlisting}

當限制被破壞時,\textit{驗證引擎}(\textit{validation engine})使用目前設定的 \texttt{Message\-Interpolator} 生成它的描述。透過 \texttt{ConstraintViolation\#getMessage()} 可以取得限制被破壞所生成的描述。

\textit{訊息參數(message parameters)} 與 \textit{訊息表示式(message expressions)} 在訊息生成的過程中會被替換成具體的訊息內容。訊息參數被放在 \texttt{\{\}} 中;訊息表示式被放在 \texttt{\$\{\}}中。

因為 \texttt{\$}, \texttt{\{} 與 \texttt{\}}是訊息描述的保留字,所以要在訊息描述中使用到這些字元需要在前面冠上反斜線 \texttt{\textbackslash},即 \texttt{\textbackslash\$}, \texttt{\textbackslash\{}, \texttt{\textbackslash\}} 以及 \texttt{\textbackslash\textbackslash} 表示反斜線。

訊息參數的內容將被當作索引(key)至 \textit{ValidationMessages} \footnote{\url{https://github.com/hibernate/hibernate-validator/tree/main/engine/src/main/resources/org/hibernate/validator}} 中查詢並替換成具體的訊息。此過程會遞迴至訊息描述不再包含訊息參數為止。開發員可以將檔案 \textit{ValidationMessages.properties} 新增至 classpath 中,以自訂訊息;亦可透過檔名後綴提供多國語系,例如 \textit{ValidationMessages\_en\_US.properties}。透過 \texttt{Locale\#getDefault()} 可得到當前 JVM 的語系。換句話說,只有將訊息描述成訊息參數才能有多國語系的功能。

訊息表示式的內容為 Jakarta Expression Language,可以提高建構訊息的彈性,例如條件式的決定訊息內容。目前被驗證的屬性將儲存在變數 \texttt{validatedValue} 中。

以下範例:

\begin{lstlisting}[numbers=left, xleftmargin=1.5\parindent]
@Data
public class Car {
  @NotNull
  private String manufacturer;

  @Size(
    min = 2,
    max = 14,
    message = "車牌 `\color{blue}'\$\{validatedValue\}'` 長度必須介於 `\color{blue}\{min\}` 與 `\color{blue}\{max\}` 長度之間"
  )
  private String licensePlate;

  @Min(
    value = 2,
    message = "There must be at least `\color{blue}\{value\}` seat`\color{blue}\$\{value > 1 ? 's' : ''\}`"
  )
  private int seatCount;

  @DecimalMax(
    value = "350",
    message = "最高速度 `\color{blue}\$\{formatter.format('\%1\$.2f', validatedValue)\}` 不能超過 `\color{blue}\{value\}`"
  )
  private double topSpeed;

  @DecimalMax(value = "100000", message = "價格不能高於 `\color{blue}\$\{value\}`")
  private BigDecimal price;
}
\end{lstlisting}

第 4 行,被標記 \texttt{@NotNull} 的 \texttt{manufacturer} 限制被破壞將產生錯誤訊息``不得是空值",它是系統預設的訊息。第 11 行,被標記 \texttt{@Size} 的 licensePlate 限制被破壞將產生帶 \texttt{min} 及 \texttt{max} 參數的訊息,而 \texttt{validatedValue} 為被驗證的車牌。第 15 行,被標記 \texttt{@Min} 的 \texttt{seatCount} 限制被破壞將產生訊息,其結尾判斷數量為複數時加上 `s'~\footnote{判斷複數名詞結尾要加 s,中文翻譯後無法表達此效果,故保留原文。},判斷式中的 \texttt{value} 變數為其限制標籤欄位值。第 24 行,被標記 \texttt{@DecimalMax} 的 \texttt{topSpeed} 限制被破壞時,將產生錯誤錯誤訊息,其中被驗證的值 \texttt{validatedValue} 被格式化處理。第 26 行,被標記 \texttt{@DecimalMax} 的 \texttt{price} 限制被破壞時,將產生訊息,其 \texttt{value} 變數的數值來至限制標籤中的 \texttt{value} 欄位。值得注意的是,第 15, 21, 25 行 使用 \texttt{\{value\}} 及 \texttt{\$\{value\}} 存取限制標籤的欄位值,這兩種都是有效的。

測試程式碼如下:

\begin{lstlisting}[basicstyle=\linespread{1}\ttfamily\scriptsize]
import static org.hibernate.validator.testutil.ConstraintViolationAssert.assertThat;
import static org.hibernate.validator.testutil.ConstraintViolationAssert.violationOf;

Car car = new Car(null, "x", 1, 400.5, BigDecimal.valueOf(100001));
Set<ConstraintViolation<Car>> violations = validator.validate(car);
assertThat(violations).containsOnlyViolations(
  violationOf(NotNull.class).withMessage("不得是空值").withProperty("manufacturer"),
  violationOf(Min.class).withMessage("There must be at least 2 seats").withProperty("seatCount"),
  violationOf(DecimalMax.class).withMessage("最高速度 400.50 不能超過 350").withProperty("topSpeed"),
  violationOf(DecimalMax.class).withMessage("價格不能高於 $100000").withProperty("price"),
  violationOf(Size.class).withMessage("車牌 'x' 長度必須介於 2 與 14 長度之間").withProperty("licensePlate")
);
\end{lstlisting}

\section{自訂生成訊息}

如果預設訊息不符合需求,以下提供數種替換訊息的方式。

\subsection{自定義 \texttt{MessageInterpolator}}

以下程式碼透過自訂的 \texttt{MessageInterpolator} 物件以自訂訊息的生成。

\begin{lstlisting}[numbers=left, basicstyle=\linespread{1}\ttfamily\scriptsize, xleftmargin=1.5\parindent]
public class CustomMessageInterpolator `\color{blue}implements MessageInterpolator` {
  private final Map<String, String> message =
    Map.of(`\color{blue}"\{jakarta.validation.constraints.NotNull.message\}"`, `\color{blue}"請提供資料"`);
  private MessageInterpolator defaultMessageInterpolator = Validation.byDefaultProvider().configure().getDefaultMessageInterpolator();

  @Override
  public String interpolate(String s, Context context) {
    if (message.containsKey(s))
      return message.get(s);
    return defaultMessageInterpolator.interpolate(s, context);
  }

  @Override
  public String interpolate(String s, Context context, Locale locale) {
    if (message.containsKey(s))
      return message.get(s);
    return defaultMessageInterpolator.interpolate(s, context, locale);
  }
}
\end{lstlisting}

第 3-4 行,將 \texttt{@NotNull} 違反限制的訊息改成 ``請提供資料"。第 7 行,接收到的參數 \texttt{s} 是訊息參數,然後第 8 行判斷其是否屬於自訂訊息,若是則在第 9 行回傳訊息內容。第 10 行,官方建議若沒自訂訊息則回傳預設 \texttt{MessageInterpolator} 的訊息。預設 \texttt{MessageInterpolator} 的取得方式參考第 4 行。

以下是測試程式碼:

\begin{lstlisting}[numbers=left, basicstyle=\linespread{1}\ttfamily\footnotesize, xleftmargin=1.5\parindent]
private Validator validator;

@BeforeMethod
public void setup() {
  ValidatorFactory factory = Validation.byDefaultProvider().configure()
    .messageInterpolator(`\color{blue}new CustomMessageInterpolator()`)
    .buildValidatorFactory();
  validator = factory.getValidator();
}

@Test
public void testAllInvalid() {
  Car car = new Car(null, "");
  Set<ConstraintViolation<Car>> violations = validator.validate(car);
  assertThat(violations).containsOnlyViolations(
    violationOf(NotNull.class).withMessage("請提供資料").withProperty("manufacturer"),
    violationOf(Size.class).withMessage("大小必須在 2 和 14 之間").withProperty("licensePlate")
  );
}
\end{lstlisting}

第 6 行套用自訂 \texttt{CustomMessageInterpolator}。在測試中,第 16 行,\texttt{@NotNull} 的限制被破壞的訊息被改成 ``請提供資料"。第 17 行,\texttt{@Size} 的限制被破壞的訊息沒有被修改,為預設內容。

\subsection{擴充預設的 \texttt{ResourceBundleMessageInterpolator}}

使用 \texttt{ResourceBundleLocator} 載入自訂的 \textit{properties} 檔案,預設為 \texttt{ResourceBundleMessageInterpolator}。

以下將 \texttt{NotNull} 限制破壞的訊息參數 \textit{jakarta.validation.constraints.NotNull.message} 的內容重新定義成 ``請提供資料",將其儲存至檔案 \textit{src/main/resources/MyMessages.properties}。

\begin{lstlisting}
jakarta.validation.constraints.NotNull.message=請提供資料
\end{lstlisting}

\begin{lstlisting}[numbers=left, basicstyle=\linespread{1}\ttfamily\footnotesize, xleftmargin=1.5\parindent]
private Validator validator;

@BeforeMethod
public void setup() {
  ValidatorFactory factory = Validation.byDefaultProvider().configure()
    .messageInterpolator(
      new ResourceBundleMessageInterpolator(`\color{blue}new PlatformResourceBundleLocator("MyMessages")`))
    .buildValidatorFactory();
  validator = factory.getValidator();
}

@Test
public void testAllInvalid() {
  Car car = new Car(null, "");
  Set<ConstraintViolation<Car>> violations = validator.validate(car);
  assertThat(violations).containsOnlyViolations(
    violationOf(NotNull.class).withMessage("請提供資料").withProperty("manufacturer"),
    violationOf(Size.class).withMessage("大小必須在 2 和 14 之間").withProperty("licensePlate")
  );
}
\end{lstlisting}

\subsubsection{多 properties 檔案}

\begin{lstlisting}
@BeforeMethod
public void setup() {
  ValidatorFactory factory = Validation.byDefaultProvider().configure()
    .messageInterpolator(new ResourceBundleMessageInterpolator(
      `\color{blue}new AggregateResourceBundleLocator(List.of("MyMessages1", "MyMessages2"))`))
    .buildValidatorFactory();
  validator = factory.getValidator();
}
\end{lstlisting}